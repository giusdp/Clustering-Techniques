\documentclass[12pt]{llncs}

\usepackage[italian]{babel}
\usepackage[utf8]{inputenc}
% \usepackage{hyperref}
\usepackage{graphicx}

\usepackage{subfig}

\usepackage[T1]{fontenc}

%\usepackage{float}

\author{Giuseppe De Palma}

\title{Progetto Elaborazione Linguaggio Naturale: Tecniche di Clustering}

\institute{Alma Mater Studiorum - Università di Bologna \\
	\email{giuseppe.depalma@studio.unibo.it}\\
	\email{Matricola: 854846}
}

\newcommand{\acapo}{\vspace{0.5\baselineskip}\\}

\begin{document}
    \maketitle
	
	\begin{abstract}
		Ciaone
	\end{abstract}
	   
	\section{Introduzione}
	Il \textit{clustering} (o analisi dei gruppi) è una forma di \textit{machine learning} non supervisionato che permette di raggruppare in \textit{clusters} elementi non annotati
	dati in input. Un cluster è una collezione di oggetti ``simili'' tra loro che sono ``dissimili'' rispetto agli oggetti degli altri cluster. Questo tipo di machine learning è
	ottimo per partizionare un insieme di dati in diverse ``categorie'', quindi poter eseguire diverse analisi ed ottenere nuove informazioni.
	Applicazioni tipiche in cui il clustering viene molto usato è il riconoscimento di email di spam (le email a scopi pubblicitari o di frode), oppure per l'aggregazione di notizie (vedasi Google News per un esempio).
	\acapo
	Il clustering trova possibili applicazioni anche nel campo dell'elaborazione del linguaggio naturale. Oltre alle nuove possibili analisi
	sui corpora ed al fornire una visualizzazione pittografica delle parole raggrupate, un interessante utilizzo è quello della \textbf{generalizzazione} delle parole.
	\acapo 
	Possiamo considerare i vari clusters delle classi di equivalenza. Per questo motivo, se avessimo un dataset su cui comporre i clusters fatto di frasi e parole, allora si potrebbe assumere che una
	qualche parola che compare in una frase può essere sostituita con un'altra dello stesso cluster lasciando intatta la correttezza della frase. 
	Ad esempio, se avessimo nel nostro dataset ``per Lunedì'', ``per Martedì'', ``per Mercoledì'', ``per Sabato'', ``per Domenica'', senza avere ``per Giovedì'' e ``per Vernedì'', e avessimo un cluster in cui 
	i giorni della settimana sono raggruppati insieme, allora potremmo generalizzare l'utilizzo della preposizione ``per'' con Giovedì e Vernedì.
	\acapo
	Il clustering, quindi, può essere molto utile anche nell'elaborazione del linguaggio naturale. Nel progetto in studio vengono testate le capacità di alcune tecniche di clustering
	da cui si derivano dei risultati per mostrarne le differenze, i pregi e i difetti. I dati utilizzati negli esperimenti, comunque, non sono parti di testo, ma semplici dataset di vettori numerici 2D in modo tale da poter facilmente 
	visualizzare i grafici relativi ai clusters e determinare le caratteristiche di ogni tecnica.
    \subsection{Outline}
	\section{Clustering}
	[PARLARE DELLE DUE CLASSI DI CLUSTERING E DI SOFT E HARD]
	\acapo
	I metodi implementati e testati sono quattro, divisi in due classi distinte:
	\begin{itemize}
		\item Clustering \textbf{gerarchico}
		\begin{enumerate}
			\item Aggregativo (o bottom-up)
			\item Divisivo (o top-down)
		\end{enumerate}

		\item Clustering \textbf{partizionale}
		\begin{enumerate}
			\item K-Means
			\item EM
		\end{enumerate}
	\end{itemize}
    \section{Sessione Sperimentale}
    \section{Conclusioni}
    
\end{document}